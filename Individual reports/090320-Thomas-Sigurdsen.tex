\documentclass[a4paper,twoside]{article}
\usepackage[hidelinks]{hyperref}
\usepackage{fancyhdr}
\usepackage{listings}
\usepackage{minted}
\usepackage{fullpage}
\usepackage[utf8]{inputenc}

\author{Thomas Sigurdsen - 090320}
\title{IMT3601 Game Programming \\ Individual Report \\ Group Neuton}
\date{\today}

\begin{document}
\headheight = 14.0pt
\pagestyle{fancy}
\maketitle
\section*{Abstract}
%% This section is to be written when all
%% else is done. 100-200 words summarizing.
This project has been refreshing and a lot of work. Even though I have had two
chunks of downtime, I feel I have given a good shot at this, and delivered an ok
product.
%\tableofcontents

\section{Features Worked On}
\subsection{Networking}
Week 50 and 51 were spent on networking and thus threading. This was all to late 
in the process. And even though I think I could have gotten the code finished,
it would have been untested and most prone to breaking at unexpected opportunities.
It still resides in the branch '30-world-network-sync'

\subsection{Configuration File}
Week 49 I spent working on reading a configuration, and creating it if it was 
missing. I spent a lot more time on this then I should have, mainly because I
wasn't familiar with the differences between 'fopen' and 'freopen'.

\subsection{Framework}
From the start of the project I did work on most of the framework. Getting among 
other things the World singleton; The MineFactory factory object-pool singleton
and Tile up and running. I also had some work doing the makefile and Includes
header.

\subsection{Libraries}
We have at times been struggling with getting libraries to play with MSBuild and
Visual Studio. I have more often than not been involved in troubleshooting these
issues as I have more experience with the workings of libraries, compilers and
linkers from my use of GNU/Linux.

\section{Things Learned}
\subsection{Libraries}
I have during this project for the first time compiled an external library. And
through that I have familiarized myself further with CMake and GNU Make.

\subsection{Git}
Learning git has been way more fun than i thought it would be. It also proved to be
really good at helping organize work, and workflow. Branching makes software 
configuration management's value increase exponentially.

\subsection{Design Patterns}
When I first learned of them at the start of the semester I was thrilled. During
this semester however, I have come to see them as tools to be catious of. Awesome
tools for the right job. Bloating if used for a wrong job.

\subsection{Threading}
Pthreads, I have learned, are both a pain and a joy. There are quite a few snags
to hit for us using (or attempting to use) object orientation and C++.

\subsection{Networking}
Is a bit of a design issue. Especially if implementing it at a late stage in 
development.

\subsection{Portability}
The biggest issue I believe we have had throughout the project is in dealing with
Visual Studio and external libraries. We have run into few problems outside of this.

\subsection{Groupmanagement}
As I did much of the work on the initial prototype and frameworks of the game, I 
got to know it rather intimately from the beginning. This lead to me sitting 
metaphorically in the middle of a lot of the decision making throughout the project.

\subsection{Generally}
\begin{description}
 \item [Static code analysis] makes me sorry I didn't use it earlier.
 \item [Valgrind] has become more of a saviour than it ever where.
 \item [Doxygen] enhances commenting. A lot.
 \item []
\end{description}





\end{document}
